\section{Simulation Analysis}
\label{sec:simulation}

In this section, the intent is to simulate the circuit, using \textit{Ngspice}. 

Firstly, the operation point was obtained and the functioning of the NPN and PNP transistors in the F.A.R was assessed, being concluded in the following tables.

\begin{center}
    \begin{tabular}{|l|r|}
      \hline    
      {\bf Variable} & {\bf Value} \\ \hline
      \input{npn_tab.tex}
    \end{tabular}
\end{center}

And

\begin{center}
    \begin{tabular}{|l|r|}
      \hline    
      {\bf Variable} & {\bf Value} \\ \hline
      \input{PNP_tab.tex}
    \end{tabular}
\end{center}

Then, time and frequency analysis were performed in order to evaluate the collector (coll) and the output (out) voltage gain in the passband, the lower ($f_1$) and the upper ($f_2$) cutoff frequencies, the bandwith (difference between the cutoff frequencies) and the input (Zin) and output (Zout) impedances of the overall circuit. The following plots and tables represent such evaluations, along with the calculation of the cost and merit figure of the circuit.

\begin{figure}[h]
    \centering
    \includegraphics[width=0.5\linewidth]{vo1.pdf}
    \caption{Magnitude db plot for voltage in node coll/collector with time}
    \label{vo1}
\end{figure}

\begin{figure}[h]
    \centering
    \includegraphics[width=0.5\linewidth]{vo1f.pdf}
    \caption{Magnitude db plot for voltage in node coll/collector with frequency}
    \label{vo1f}
\end{figure}

\begin{figure}[h]
    \centering
    \includegraphics[width=0.5\linewidth]{vo2f.pdf}
    \caption{Magnitude db plot for voltage in node out/output with frequency}
    \label{vo2f}
\end{figure}

\clearpage

The tables are 

\begin{center}
    \begin{tabular}{|l|r|}
      \hline    
      {\bf Variable} & {\bf Value} \\ \hline
      \input{measurements_tab.tex}
    \end{tabular}
\end{center}

\begin{center}
    \begin{tabular}{|l|r|}
    \hline    
    {\bf Variable} & {\bf Value} \\ \hline
    \input{Zout_tab.tex}
    \end{tabular}
\end{center}

The parameters aimed to optimize are in the following table once more.

\begin{center}
    \begin{tabular}{|l|r|}
      \hline    
      {\bf Variable} & {\bf Value} \\ \hline
      \input{goals_tab.tex}
    \end{tabular}
\end{center}

Understanding the purpose of the coupling capacitors on our circuit and confirming that understanding by "playing around" (changing their parameter values) on ngspice, it can be infered that the bandwith increases as the capacitance increases, since the lower cutoff frequency ($f_1$) decreases and the higher one ($f_2$) isn't affected. This increase will allow the transistors to keep operating at lower frequencies, since their impedance tends to infinity with $\omega$ approaching 0.

Understanding the purpose of the bypass capacitor on our circuit and confirming that understanding by "playing around" (changing its parameter value) on ngspice, it can be infered that, as its capacitance increases, the first stage gain increases. That is because its impedance is $1/(jwC)$ ($w=2 \pi f$), and it is placed in paralel with a resistance ($R_E$), which varies inversely with the first stage gain. As such, with higher frequencies and capacitance, the impedance diminishes and, gradually, the capacitor starts to work as a short-circuit and to bypass the resistor $R_E$.

Understanding the impact of the $R_C$ on our circuit and confirming that understanding by "playing around" (changing its parameter value) on ngspice, it can be infered that, as $R_C$ increases, the gain also increases, which can be greatly observed with a theorical analysis of the circuit asweel: $R_C$ and the gain of the circuit are proportional.

